% =========================
% macros.tex (hardened)
% =========================

\newcommand{\swatch}[1]{\fcolorbox{black}{#1}{\rule{1em}{1em}}}

% --- Safe fallback if hyperref isn't loaded yet ---
\providecommand{\texorpdfstring}[2]{#1}

% --- Robust Ishtar macro (works in headings/TOC/PDF) ---
\DeclareRobustCommand{\ishtar}{\texorpdfstring{\textbf{Ishtar\,AI}}{Ishtar AI}}

% --- TikZ styles
% NOTE: For the 'cylinder' shape, ensure in the preamble:
%   \usetikzlibrary{shapes.symbols}
% If not available, we fall back to a rounded rectangle.
\makeatletter
\@ifundefined{pgfdeclareshape}{%
  % PGF/TikZ not loaded yet: define a minimal, safe placeholder
  \newcommand{\pgfshapecylinderfallback}{}
  \tikzset{db/.style={draw, rounded corners, minimum height=10mm, minimum width=12mm}}
}{%
  % PGF/TikZ present; try to use cylinder, otherwise fallback
  \@ifundefined{pgfdeclareshape@cylinder}{%
    \tikzset{db/.style={draw, rounded corners, minimum height=10mm, minimum width=12mm}}
  }{%
    \tikzset{db/.style={cylinder, draw, shape aspect=0.3, minimum height=10mm, minimum width=10mm}}
  }%
}
\makeatother

% --- Global TikZ defaults ---
\tikzset{
  >=Stealth,
  every picture/.append style={line width=0.6pt},
  every node/.append style={font=\footnotesize},
}

% --- TikZ baseline styles (keep diagrams consistent) ---
\tikzset{
  box/.style={draw, rounded corners, align=center, inner sep=4pt, minimum height=10mm},
  arrow/.style={-Stealth, thick},
  lane/.style={draw, rounded corners, inner sep=6pt}
}

% --- Column type 'Y' only if array/tabularx is available ---
\makeatletter
\@ifundefined{newcolumntype}{%
  % 'array' not loaded yet; defer defining Y to where array/tabularx is loaded
}{%
  \newcolumntype{Y}{>{\raggedright\arraybackslash}X}%
}
\makeatother

% --- Inline/formatting helpers (use unique or guarded names) ---
% If \header already exists, overwrite it; else define it.
\makeatletter
\@ifundefined{header}{%
  \newcommand{\header}[1]{\textbf{#1}}%
}{%
  \renewcommand{\header}[1]{\textbf{#1}}%
}
\makeatother

% Define \attr only if it's not defined elsewhere
\providecommand{\attr}[1]{\texttt{#1}}

% --- Optional product/term wrappers (kept commented if unused) ---
% \newcommand{\product}[1]{\textsf{#1}}
% \newcommand{\term}[1]{\textit{#1}}
% \newcommand{\code}[1]{\texttt{#1}}

% ----------------------------
% Standard callout boxes
% ----------------------------
\newcommand{\BestPracticeBox}[1]{%
  \begin{tcolorbox}[
    title={\textbf{Best Practice}},
    colback=green!5,
    colframe=green!40!black,
    colbacktitle=green!20,
    coltitle=black,
    fonttitle=\bfseries,
    boxrule=0.7pt,
    arc=4pt,
    left=5mm, right=5mm, top=4mm, bottom=4mm,
    before skip=6pt,
    after skip=6pt
  ]
  #1
  \end{tcolorbox}%
}
\newcommand{\PitfallBox}[1]{%
  \par\addvspace{12pt}%
  \begin{svgraybox}\textbf{Pitfall.} #1\end{svgraybox}%
  \par\addvspace{12pt}%
}
\newcommand{\DefinitionBox}[1]{%
  \par\addvspace{12pt}%
  \begin{svgraybox}\textbf{Definition.} #1\end{svgraybox}%
  \par\addvspace{12pt}%
}
\newcommand{\ChecklistBox}[2][Checklist]{%
  \begin{tcolorbox}[
    title={\textbf{#1}},
    colback=blue!5,
    colframe=blue!40!black,
    colbacktitle=blue!20,
    coltitle=black,
    fonttitle=\bfseries,
    boxrule=0.7pt,
    arc=4pt,
    left=5mm, right=5mm, top=4mm, bottom=4mm,
    before skip=12pt,
    after skip=12pt,
    breakable
  ]
  #2
  \end{tcolorbox}%
}
\newcommand{\IshtarVignette}[1]{%
  \par\addvspace{12pt}%
  \begin{svgraybox}\textbf{Ishtar AI Vignette.} #1\end{svgraybox}%
  \par\addvspace{12pt}%
}

% --- Listings default style (production-safe) ---
\lstdefinestyle{springer}{
  basicstyle=\ttfamily\small,
  columns=fullflexible,
  breaklines=true,
  breakatwhitespace=true,
  numbers=left,
  numberstyle=\tiny,
  stepnumber=1,
  numbersep=6pt,
  frame=single,
  framerule=0.2pt,
  xleftmargin=1.5em,
  tabsize=2,
  showstringspaces=false
}
\lstset{style=springer}

% ==========================================================
% Figure fitting helpers (SNmono-safe)
% ==========================================================
% Maximum figure dimensions: allows figures to be 15% wider than linewidth
% and up to 90% of page height. These values can be adjusted per-figure
% if needed by redefining \LLMFigMaxWidth or \LLMFigMaxHeight locally.
\newcommand{\LLMFigMaxWidth}{1.15\linewidth}
\newcommand{\LLMFigMaxHeight}{0.9\textheight}

% A wrapper that scales DOWN to fit the Springer textblock,
% and prevents overly-tall diagrams from spilling off the page.
% Note: adjustbox scales content proportionally, so increasing TikZ sizes
% will be visible up to these maximum constraints.
\newenvironment{llmfigbox}{%
  \begin{adjustbox}{max width=\LLMFigMaxWidth, max totalheight=\LLMFigMaxHeight, center}%
}{%
  \end{adjustbox}%
}

% A wrapper that centers content without scaling constraints.
% Use this when you want full control over figure sizing.
\newenvironment{llmfigboxnoscale}{%
  \begin{adjustbox}{center}%
}{%
  \end{adjustbox}%
}
